% -*- TeX:de -*-
\NeedsTeXFormat{LaTeX2e}
\documentclass[12pt,a4paper]{article}
\usepackage[english]{babel} % german text
\usepackage[DIV12]{typearea} % size of printable area
\usepackage[T1]{fontenc} % font encoding
%\usepackage[latin1]{inputenc} % most likely on Windows
\usepackage[utf8]{inputenc} % probably on Linux
\usepackage{multicol}
% PLOTTING
\usepackage{pgfplots}
\usepackage{pgfplotstable}
\usepackage{url}
\usepackage{graphicx} % to include images
\usepackage{tikz}
\usepackage{subfigure} % for creating subfigures
\usepackage{amsmath} % a bunch of symbols
\usepackage{amssymb} % even more symbols
\usepackage{booktabs} % pretty tables
\usepackage{makecell} % multi row table heading
% a floating environment for circuits
\usepackage{float}
\usepackage{caption}
\usepackage{hyperref}

% Title Page command
\newcommand{\HRule}{\rule{\linewidth}{0.5mm}}

%\newfloat{circuit}{tbph}{circuits}
%\floatname{circuit}{Schaltplan}
% a floating environment for diagrams
%\newfloat{diagram}{tbph}{diagrams}
%\floatname{diagram}{Diagramm}
\pgfplotsset{compat=1.8}
\selectlanguage{english} % use german
\begin{document}
%%%%%%% DECKBLATT %%%%%%%
\begin{titlepage}
\begin{center}

% Upper part of the page. The '~' is needed because \\
% only works if a paragraph has started.
\includegraphics[scale=0.75]{./unilogo}~\\[2cm]

\textsc{\LARGE University of Vienna }\\[0.5cm]
\textsc{\LARGE Faculty of Physics}\\[1.5cm]
\textsc{\Large Bachelor Thesis}\\[0.5cm]

% Title
\HRule \\[0.4cm]
{ \huge \bfseries Solving the Traveling Salesman Problem with tree algorithms}\\[0.4cm]

\HRule \\[1.5cm]

% Author and supervisor
\begin{minipage}{0.4\textwidth}
\begin{flushleft} \large
\emph{Author:}\\
Johannes \textsc{Kurz}
\end{flushleft}
\end{minipage}
\begin{minipage}{0.4\textwidth}
\begin{flushright} \large
\emph{Supervisor:} \\
Univ.-Prof. Dr. Christoph \textsc{Dellago}
\end{flushright}
\end{minipage}

\vfill

% Bottom of the page
{\large \today}

\end{center}
\end{titlepage}
%%%%%%% DECKBLATT ENDE %%%%%%%
\pagebreak
\setlength{\columnsep}{20pt}
\begin{multicols}{2}

\begin{abstract}

\end{abstract}

%%%%%%%%%%%%%%%%%%%%%%%%%%%%%%%%%%%%%%%%%%%%%%%%
%\begin{figure}[H]
% \centering
% \includegraphics[scale=0.35]{./data/beugung.png}
% \caption{Beugungsmuster Einzelspalt (echtes Foto; schwarz durch weiß ersetzt)}
% \label{fig:beugungsmuster}
%\end{figure}
%\begin{figure}[H]
% \centering
% \pgfplotstabletypeset[
% columns={abstand, n},
% col sep=&,
% columns/abstand/.style={precision=2, zerofill, column name=\makecell{$Abstand$\\$(\pm 0.05)[mm]$} },
% columns/n/.style={column name=\makecell{$n$\\$(Ordnung)$}, precision=0},
% every head row/.style={before row=\hline,after row=\hline\hline},
% every last row/.style={after row=\hline},
% every first column/.style={column type/.add={|}{} },
% every last column/.style={column type/.add={}{|} }
% ]{
% abstand & n
% 12.9 & 1
% 24.45 & 2
% 37.40 & 3
% 49.35& 4
% 62.45 & 5
% 74.45 & 6
% 87.45 & 7
% 100.25 & 8
%
% }
% \caption{Messwerte Einzelspalt}
% \label{tab:werte_einzelspalt}
%\end{figure}
%%%%%%%%%%%%%%%%%%%%%%%%%%%%%%%%%%%%%%%%%%%%%%%%
%%%%%%%%%%%%%%%%%%%%%%%%%%%%%%%%%%%%%%%%%%%%%%%%

\section{Radiation pressure}
Goal of this work is to demonstrate how massless photons counter intuitively create pressure based on quantum mechanical
properties.
On the following pages the theoretical basics will be explained and enhanced to fit the experiment.
The Experiment itself will be explained in further sections. Results and discussion about the outcome
will complete this work.

%%%%%%%%%%%%%%%%%%%%%%%%%%%%%%%%%%%%%%%%%%%%%%%%
%%%%%%%%%%%%%%%%%%%%%%%%%%%%%%%%%%%%%%%%%%%%%%%%
\section{Theory}
\label{theory}
The foundation of photons inducing a pressure comes from the photoelectric effect where (Max Planck) stated that:
$$E = h * \nu$$
The energy E of a photon is proportional to the Planck constant times the frequency $\nu$.
From this knowledge combined with (Einstein's) law $E = m * c^2$ we can derive a mass:
$$m = \frac{h * \nu}{c^2}$$
Combined with the classical definition of momentum $p = m * v$ we get:
$$p = \frac{h * \nu}{c^2} * c = \frac{h * \nu}{c}$$
Due to the fact that pressure is defined as force per square meter we can get the pressure by the time derivative of the momentum times the number of particles per area element. The outcome is a simple dependency between force and power of the laser:
$$F_0 = \frac{2P}{c}$$

\subsection{Cantilever}
A cantilever is a high (> 99\%) reflective mirror on top of a long rod so it can oscillate back and forth.
High reflectivity is needed because the highest force transfer and therefore pressure is obtained when all photons are reflected. We have a set of multiple cantilevers on a single plate \ref{cantilever}.
\begin{figure}[H]
	\centering
	\includegraphics[scale=0.2]{../figures/cantilever.png}
	\caption{Multiple cantilever on one plate \cite{physikwiki}}
	\label{fig:cantilever}
\end{figure}
\subsection{Dampend driven harmonic oscilliation}
With a pulsed laser and a previous described cantilever we can build a oscillator that responds to the force of the photons. Because we want to measure a very tiny pressure the cantilever has to be driven by the pulsed laser at the resonance frequency of the oscillator. Due to the fact that we don't cool or evacuate our setup we have a dampening that is overcome by the repeated pulsing of the laser with the resonance frequency.\\
The calculation of the (mechanical) oscillator is stated in the experimental description \cite{physikwiki}.\\
Properties of the oscillator are the resonance frequency and the quality factor (taken from \cite{physikwiki}).\\
The quality factor:
$$Q = \frac{\omega_0}{\gamma}$$
Resonanz frequency :
$$\omega_0 = \sqrt{\frac{k}{m}}$$

%%%%%%%%%%%%%%%%%%%%%%%%%%%%%%%%%%%%%%%%%%%%%%%%
%%%%%%%%%%%%%%%%%%%%%%%%%%%%%%%%%%%%%%%%%%%%%%%%
\section{Experimental assembly}
We have a build up table top experiment composed of different components which can be seen in \ref{fig:setup}.\\
\begin{figure}[H]
	\centering
	\includegraphics[scale=0.2]{../figures/aufbau.png}
	\caption{Setup}
	\label{fig:setup}
\end{figure}
% TODO Aufbau Beschreiben

%% TODO Akustische Messung, Displacement bei pulsed Laser

\begin{figure}[H]
	\centering
	\includegraphics[scale=0.5]{../figures/messung.png}
	\caption{Measurment}
	\label{fig:measurement}
\end{figure}

\noindent
\textbf{How to interpret the measurement}\\
Now combine the theory from \ref{theory} into one formula with the knowledge about the oscillator in \cite{physikwiki}:
$$\gamma = \frac{D}{m} = \frac{\omega_0}{Q}$$
From the solution to the harmonic driven dampened oscillator in \cite{physikwiki} we derive a formula for the position:
$$x_0 = \frac{F_0}{m}  \frac{1}{\sqrt{ (\omega_0^2 - \omega^2 )^2 + \gamma^2  \omega^2}}$$
Because the interesting point is where $\omega_0 = \omega$ (resonance) we can set that to zero and get
$$x_0 = \frac{F_0}{m} * \frac{1}{\gamma  \omega}$$
using the definition for $\gamma$ and $\omega_0^2$ we get
$$k = \frac{F_0 * Q}{x_0}$$
for the spring constant.\\
The $x_0$ is what we measure depending on the intensity of the laser, the displacement on the sensor, the distance of the beam from cantilever to sensor and the length of the cantilever (resulting in a conversion factor from sensor result to displacement).

%%%%%%%%%%%%%%%%%%%%%%%%%%%%%%%%%%%%%%%%%%%%%%%%
%%%%%%%%%%%%%%%%%%%%%%%%%%%%%%%%%%%%%%%%%%%%%%%%
\section{Results}
The full document and the results contained in a QTI file and a Excel sheet can be found in \cite{github}.

% TODO werte!

\begin{figure}[H]
	\centering
	\includegraphics[scale=1.2]{../figures/Resonanzkurve.png}
	\caption{Resonance curve}
	\label{fig:resonanzkurve}
\end{figure}

\begin{figure}[H]
	\centering
	\includegraphics[scale=1.4]{../figures/ResonanzkurveHalbwertsbreite.png}
	\caption{Full width half maximum of the resonance curve}
	\label{fig:resonanzkurvehmfuw}
\end{figure}

\begin{figure}[H]
	\centering
	\includegraphics[scale=1.4]{../figures/Slopeofsensitivityofthesensor.png}
	\caption{Slope of the sensitivity of the sensor}
	\label{fig:sensorsensitivity}
\end{figure}


%%%%%%%%%%%%%%%%%%%%%%%%%%%%%%%%%%%%%%%%%%%%%%%%
%%%%%%%%%%%%%%%%%%%%%%%%%%%%%%%%%%%%%%%%%%%%%%%%
\section{Discussion}

% TODO lob und Kritik

%%%%%%%%%%%%%%%%%%%%%%%%%%%%%%%%%%%%%%%%%%%%%%%%
%%%%%%%%%%%%%%%%%%%%%%%%%%%%%%%%%%%%%%%%%%%%%%%%
\pagebreak

\bibliography{protocol.bib}
\bibliographystyle{plain}

\end{multicols}
\end{document}