% -*- TeX:de -*-
\NeedsTeXFormat{LaTeX2e}
\documentclass[12pt,a4paper]{article}
\usepackage[english]{babel} % german text
\usepackage[DIV12]{typearea} % size of printable area
\usepackage[T1]{fontenc} % font encoding
%\usepackage[latin1]{inputenc} % most likely on Windows
\usepackage[utf8]{inputenc} % probably on Linux
\usepackage{multicol}
% PLOTTING
\usepackage{pgfplots}
\usepackage{pgfplotstable}
\usepackage{url}
\usepackage{graphicx} % to include images
\usepackage{tikz}
\usepackage{subfigure} % for creating subfigures
\usepackage{amsmath} % a bunch of symbols
\usepackage{amssymb} % even more symbols
\usepackage{booktabs} % pretty tables
\usepackage{makecell} % multi row table heading
% a floating environment for circuits
\usepackage{float}
\usepackage{caption}
\usepackage{hyperref}

% Title Page command
\newcommand{\HRule}{\rule{\linewidth}{0.5mm}}

%\newfloat{circuit}{tbph}{circuits}
%\floatname{circuit}{Schaltplan}
% a floating environment for diagrams
%\newfloat{diagram}{tbph}{diagrams}
%\floatname{diagram}{Diagramm}
\pgfplotsset{compat=1.8}
\selectlanguage{english} % use german


\begin{document}
%%%%%%% DECKBLATT %%%%%%%
\begin{titlepage}
\begin{center}

% Upper part of the page. The '~' is needed because \\
% only works if a paragraph has started.
\includegraphics[scale=0.75]{./unilogo}~\\[2cm]

\textsc{\LARGE University of Vienna }\\[0.5cm]
\textsc{\LARGE Faculty of Physics}\\[1.5cm]
\textsc{\Large Bachelor Thesis}\\[0.5cm]

% Title
\HRule \\[0.4cm]
{ \huge \bfseries Solving the Traveling Salesman Problem with tree algorithms}\\[0.4cm]

\HRule \\[1.5cm]

% Author and supervisor
\begin{minipage}{0.4\textwidth}
\begin{flushleft} \large
\emph{Author:}\\
Johannes \textsc{Kurz}
\end{flushleft}
\end{minipage}
\begin{minipage}{0.4\textwidth}
\begin{flushright} \large
\emph{Supervisor:} \\
Univ.-Prof. Dr. Christoph \textsc{Dellago}
\end{flushright}
\end{minipage}

\vfill

% Bottom of the page
{\large \today}

\end{center}
\end{titlepage}
%%%%%%% DECKBLATT ENDE %%%%%%%
\pagebreak
\setlength{\columnsep}{30pt}


\begin{abstract}
\noindent In this experiment, a helium-neon-laser is built and characterized. We describe the building process, from a set of available predefined components, as well as problems and their workarounds.\\
Furthermore polarization, shape of the beam and free spectral range are measured in order to calculate the characterizing parameters of the Gaussian beam.


\end{abstract}

\begin{multicols}{2}

%%%%%%%%%%%%%%%%%%%%%%%%%%%%%%%%%%%%%%%%%%%%%%%%
%\begin{figure}[H]
% \centering
% \includegraphics[scale=0.35]{./data/beugung.png}
% \caption{Beugungsmuster Einzelspalt (echtes Foto; schwarz durch weiß ersetzt)}
% \label{fig:beugungsmuster}
%\end{figure}



%\begin{figure}[H]
% \centering
% \pgfplotstabletypeset[
% columns={abstand, n},
% col sep=&,
% columns/abstand/.style={precision=2, zerofill, column name=\makecell{$Abstand$\\$(\pm 0.05)[mm]$} },
% columns/n/.style={column name=\makecell{$n$\\$(Ordnung)$}, precision=0},
% every head row/.style={before row=\hline,after row=\hline\hline},
% every last row/.style={after row=\hline},
% every first column/.style={column type/.add={|}{} },
% every last column/.style={column type/.add={}{|} }
% ]{
% abstand & n
% 12.9 & 1
% 24.45 & 2
% 37.40 & 3
% 49.35& 4
% 62.45 & 5
% 74.45 & 6
% 87.45 & 7
% 100.25 & 8
%
% }
% \caption{Messwerte Einzelspalt}
% \label{tab:werte_einzelspalt}
%\end{figure}

%%%%%%%%%%%%%%%%%%%%%%%%%%%%%%%%%%%%%%%%%%%%%%%%
%%%%%%%%%%%%%%%%%%%%%%%%%%%%%%%%%%%%%%%%%%%%%%%%

%\section{Helium-Neon-Laser}



%%%%%%%%%%%%%%%%%%%%%%%%%%%%%%%%%%%%%%%%%%%%%%%%
%%%%%%%%%%%%%%%%%%%%%%%%%%%%%%%%%%%%%%%%%%%%%%%%
\section{Theory}
 %\cite{physikwiki}.
A $laser$ is a light source, distinguished by its spatial and temporal coherence.\\
Spatial coherence means, the (usually very narrow) beam preserves it's profile over long distances (for example allowing for the precise dot produced by a laser pointer).\\
Temporal coherence shows in a very narrow frequency-spectrum of the beam (ideally monochromatic) and its polarization.\\

\subsection{Working principle}
\noindent Since a laser produces its output by "amplifying light through stimulated emission", it is built from some sort of \textbf{gain medium} (he-ne-mixture in this experiment) that is excited: This process is called \textbf{pumping} and is usually accomplished by another light-source or, as in this case, by an applied voltage.\\
The atoms of the he-ne-gas get excited to a higher energetic state. This state is not random, but highly correlated to the frequency of the e-field generated by the voltage, hence most of the excited atoms are in the same (or a few frequency-wise neighboring states). If an atom transitions back to its lower energy state, it emits a photon of the energy-difference. These photons can also be absorbed by other atoms and excite them. Since the probability of these transition is greatly increased by the pumping-process, this is called \textbf{stimulated emission}.
If the system is in a state of \textit{population inversion}, meaning that more atoms are in an excited state than are in ground state, more photons are emitted than absorbed and the light gets amplified.\\
\\
The gain medium is placed in an optical resonator (or 'cavity'), that consists of 2 mirrors, 1 that reflects (ideally) all photons back into the medium and another one with a (small) transmittance.\\
The photons bounce back and forth between the 2 mirrors and possibly excite the atoms again, thus producing a great amount of photons with same energy and impulse, leading to a coherent beam of photons leaving through the slightly transmissive mirror.\\

\noindent The pumping energy has to be large enough to overcome losses because of photons of other frequencies (that do not resonate) and material imperfections.\\

\subsection{Characterization and Measurements:}

\textbf{Polarization:}\\
The laser-beam should be highly linear polarized. Part of this experiment is to quantify the polarization by determining its visibility:
$$V = \frac{P_{max}-P_{min}}{P_{max}+P_{min}}$$
where $P_{max}$ and $P_{min}$ are the minimum (or maximum) intensities that pass through a polarizing beam-splitter (PBS) after a half-wave-plate is applied in different angles.\\

\noindent \textbf{Beam-Shape:}\\
The beam emitted from the he-ne-laser is approximately a Gaussian-Beam. The intensity-distribution in the beam-profile follows a Gauss-distribution, and the beam-waist expands at a narrow angle.\\
The shape of the beam can thus be characterized by 2 parameters: the smallest beam-waist $\omega_0$ and the angle of its divergence. Furthermore we will try to verify a nearly circular profile by looking at the (gaussian) intensity distribution of the profile in 2 axis.\\

\noindent \textbf{Free spectral range - FSR}\\
The FSR is the difference in frequency between the modes of an optical cavity. There's resonance for waves of which the wavelength times an integer equals cavity length (these are called modes or standing waves).\\
So the FSR is a property of the cavity and dependent on its length.
$$\nu_{FSR} = \frac{c}{2L}$$
c... speed of light \\
L... cavity-length\\


\noindent The photons emitted from the atoms in the gas have certain preferred wavelengths, but not the sharp spectrum of the laser-beam. These wavelength cover about 3 to 4 modes of the cavity, resulting in a beam of (nearly) only those frequencies.\\
Since it is difficult in this experimental setup to measure the cavity-length precise enough, we will also apply 2 other methods, using the 3 wavelengths of the self-built laser:\\

\noindent One method makes use of a Fabry-Perot-Interferometer, a cavity similar to the laser's, where the position of one mirror is changeable by applying voltage to a piezo-crystal. By sweeping through different lengths of the reference cavity, one measures only then sharp intensity peaks, when its resonance matches an emitted mode of the laser.\\
Knowing the FSR of the reference cavity, one can determine the FSR of the laser-cavity. The velocity of the sweep has to be reasonable fast to not change the FSR of the measurement-cavity in the process.\\

\noindent A 3rd method is to measure the beat signal between 2 emitted wavelengths in a spectrum-analyzer. Since the frequencies themselves are to high and close together to measure their difference directly, one can make use of the much slower beat-signal produced by the superposition of different modes.\\

\noindent \textbf{Cavity Finesse:}
The cavity-finesse quantifies the quality of a cavity, it's the ratio between the FSR and the width of a single resonance-peak. Since it also matches the mean number of oscillations the photons do, before leaving the cavity, it can also be calculated from the reflectivity of the transmitting mirror (assuming the other one has a reflectivity of 1).
$$F = \frac{\pi \cdot R}{\sqrt{1 - R}} = \frac{\nu_{FSR}}{\Delta \nu}$$
F... Finesse\\
R...reflectivity\\
$\Delta \nu$ ... width of cavity resonance\\

\noindent \textbf{Stability-criterion of the cavity:}\\
In reality no alignment could be as perfect as to let the beam bounce between to parallel straight mirrors, so curved mirrors have to be used. This places a restriction on the cavity length:
$$0 \le g_1*g_2 \le 1$$
$$g_{1,2} = 1 - \frac{d}{R_{1,2}}$$
where\\
d ... cavity-length\\
$R_{1,2}$ ... radius of curvature of the mirrors\\

%%%%%%%%%%%%%%%%%%%%%%%%%%%%%%%%%%%%%%%%%%%%%%%%
%%%%%%%%%%%%%%%%%%%%%%%%%%%%%%%%%%%%%%%%%%%%%%%%
\section{Experimental Setup}
\subsection{Building the Laser}
The whole setup is built on an optical table (breadboard) and matching optical elements (mostly Thorlabs).\\
The laser itself consists of a gas-tube (filled with an helium-neon-mixture) with electrical contacts on both ends to apply high-voltage, and a slightly transmissive mirror on an adjustable mirror-mount. The fully reflective mirror is already built-in the tube.
For the construction of the laser we used an additional laser (alignment-laser) and 2 mirrors on adjustable mounts to couple into the cavity.\\
One of the 2 mirrors is placed quite far away from the gas-tube, so that changes of its angle result in translational shifts of the laser-point on the cavity-mirror. The second one is placed fairly near the cavity to produce mainly an angular shift, thus dividing the 4 degrees of freedom (2 angular, 2 translational).\\
It's recommended to adjust the alignment-laser so that the beam gets reflected back into the laser in addition to passing through the whole tube. The longer beam-path back gives a finer measurement of the correct angle, than just the pass-through-requirement.\\

\noindent After the pre-alignment, the second cavity-mirror is placed at the cavity-output. The distance to the other mirror has to fulfill the stability-requirement of the cavity and should be es orthogonal as possible to the beam-path.\\
After applying voltage from a power-supply via the 2 contacts on the tube, the gas should start to glow and the laser possibly already works. Usually this is not the case, but one should 'search' for the right angle of the curved mirror.\\
If there's no laser-beam after tilting the mirror, then the pre-alignment has to be checked, improved or done again. Also different cavity-lengths can be tried:\\
The shorter, the safer, but there's a possibility of arc discharge, if the mirror-mount is too close to the contacts.

\subsection{Measurements}
\textbf{Polarization:}\\
The Polarization is measured by placing first a rotatable half-wave-plate ($\lambda$/2-plate) and behind a polarizing beam-splitter into the beam-path. A power-meter is used to measure the intensity behind the PBS, either transmitted or reflected. The $\lambda$/2-plate changes the direction of the polarization and for different rotations (here every 10$^\circ$) the intensity after the PBS is measured. The resulting curve should oscillate between 2 intensities and after a sine-fit $P_{max}$ and $P_{min}$ can be used to calculate the visibility of the Polarization.\\

\noindent \textbf{FSR:}\\
For the free-spectral-range-measurement by cavity length, a simple ruler is used to measure the length, thus being very imprecise (the exact position of the mirror-surfaces is hidden in the mounts).\\
To couple into the Fabry-Perot-interferometer as well as into the optical fiber leading to the diode

%%%%%%%%%%%%%%%%%%%%%%%%%%%%%%%%%%%%%%%%%%%%%%%%
%%%%%%%%%%%%%%%%%%%%%%%%%%%%%%%%%%%%%%%%%%%%%%%%
\section{Results}
%\cite{github}


%%%%%%%%%%%%%%%%%%%%%%%%%%%%%%%%%%%%%%%%%%%%%%%%
%%%%%%%%%%%%%%%%%%%%%%%%%%%%%%%%%%%%%%%%%%%%%%%%
\section{Discussion}


%%%%%%%%%%%%%%%%%%%%%%%%%%%%%%%%%%%%%%%%%%%%%%%%
%%%%%%%%%%%%%%%%%%%%%%%%%%%%%%%%%%%%%%%%%%%%%%%%
%\bibliography{protocol.bib}
%\bibliographystyle{plain}
\end{multicols}
\end{document}